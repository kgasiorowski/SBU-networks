\documentclass[12pt]{report}
%\usepackage[fleqn]{amsmath}
\usepackage{amsmath}
\usepackage{ragged2e}
\setlength\parindent{0pt}

\begin{document}

\Large
\centering
CSE310 Written Homework \#1

\justify
\normalsize

Kuba Gasiorowski\\
ID: 109776237\\

\noindent1a) Time spent to get to first router: $t = \frac{8\;Mb}{2\;Mb/s} =\boxed{4s}$\\	
Time spent in total: $4s \cdot 3\;links = \boxed{12s}$\\

\noindent1b) $\frac{1\cdot10^4\;bits}{2\cdot10^6\;bits/s} = \frac{1}{2\cdot10^2} s = \boxed{0.005s}$\\
The second packet will be fully received at the first switch at time $t=$\fbox{$0.01s$}\\

\noindent1c) 
\begin{align*}
t &= \frac{(n+p+1)L}{R}\\
t &= \frac{(3+800+1) \cdot 10^4\;bits}{2\cdot10^6\;bits/s}\\
t &= \frac{8.04\cdot10^6\;bits}{2 \cdot 10^6\;bits/s}\\
t &= \boxed{4.02s}
\end{align*}
By utilizing message segmentation, the time taken to transmit the message is lowered by a factor of 3. This is because we are fully utilizing all links in the network as much as possible, instead of using one link at a time.\\

\noindent2a) $R = 2 \cdot 10^6\;bits/s$\\
$d_{prop} = \frac{link\;length}{propagation\;speed} = \frac{2\cdot10^7\;m}{2.5 \cdot 10^8\; m/s} = \frac{2}{25}\;s = \boxed{0.08\;s}$\\

\noindent$R \cdot d_{prop} = 0.08 \cdot 2 \cdot 10^6 = 0.16 \cdot 10^6 = \boxed{1.6 \cdot 10^5}$\\

\noindent2b) When the first packet is sent through the link, it spends 0.08 seconds in the link. During this 0.08 seconds, the link continues to accept bits at a rate equal to the bandwidth (2Mbps). So the total number of bits in the link is $0.08s \cdot 2\cdot10^6b/s =$ \fbox{$1.6 \cdot 10^5\;bits$}.\\

\noindent2c) The bandwidth-delay product is the maximum number of bits that can be in the link at any given time.\\
\\
2d)$w_{bit} = \frac{m}{R \cdot d_{prop}} = \frac{2 \cdot 10^7\;m}{1.6\cdot10^5\;m/b} = \frac{200}{1.6} = \boxed{125\;m}$\\
If a bit is 125m wide, then a bit is a little more than twice as wide as a football field.\\
\\
2e) $w_{bit} = \frac{m}{R\cdot{}d_{prop}} = \boxed{\frac{m}{R \cdot \frac{m}{s}}}$\\

\noindent3)
\begin{align*}
F &= 1.5 \cdot 10^{10}\;bits\\
u_s &= 3.0 \cdot 10^7\;bits/s\\
d_i &= 2.0 \cdot 10^6\;bits/s
\end{align*}

\centering

For client-server scenario: $D_{cs} = max\left(\frac{N\cdot{}F}{u_s}, \frac{F}{d_i}\right)$\\
\medskip
$N$, the number of users, is represented on the left hand side, has values of 10, 100, and 1000. $u$, the upload speed, has values of $3\cdot10^5$, $7\cdot10^5$, and $2\cdot10^6$ and is represented on the bottom. All units in the table are seconds.

\begin{table}[h]
	\centering
	\begin{tabular}{l|l|l|l|}
		\hline
		\multicolumn{1}{|l|}{10}  & 7500                    & 7500                    & 7500                    \\ \hline
		\multicolumn{1}{|l|}{100} & 50000                   & 50000                   & 50000                   \\ \hline
		\multicolumn{1}{|l|}{1000} & 500000 & 500000 & 500000 \\ \hline
		& $3\cdot10^5$ & $7\cdot10^5$ & $2\cdot10^6$ \\ \cline{2-4} 
	\end{tabular}
\end{table}
\hrulefill\\
\medskip
For peer-to-peer scenario: $D_{p2p} = max\left(\frac{F}{u_s}, \frac{F}{d_i}, \frac{N\cdot{}F}{\sum{u_i}}\right)$

Again, $N$ is represented on the left and $u$ at the bottom.

\begin{table}[h]
	\centering
	\begin{tabular}{l|l|l|l|}
		\hline
		\multicolumn{1}{|l|}{10}  & 500000                    & 214285.71                    & 75000                    \\ \hline
		\multicolumn{1}{|l|}{100} & 5000000                   & 2142857.14                   & 750000                   \\ \hline
		\multicolumn{1}{|l|}{1000} & 50000000 & 21428571.43 & 7500000 \\ \hline
		& $3\cdot10^5$ & $7\cdot10^5$ & $2\cdot10^6$ \\ \cline{2-4} 
	\end{tabular}
\end{table}

\end{document}