\documentclass[12pt]{report}
\usepackage{amsmath}
\usepackage{ragged2e}
\setlength\parindent{0pt}

\begin{document}

\Large
\centering
CSE310 Written Homework \#2

\justify
\normalsize

Kuba Gasiorowski\\
ID: 109776237\\

\noindent{}1a) $2^{8\cdot4} = 2^{32} = 4,294,967,296$ possible sequence numbers. MSS is irrelevant since the sequence number is incremented by the number of bytes sent. So, $\boxed{L = 4,294,967,296\;B}$

\bigskip

\noindent{}1b) The number of segments $S = \frac{2^{32}\;B}{MSS} = \frac{2^{32}\;B}{512\;B} = 8,388,608\;segments$\\
Since each segment is 512 bytes of payload and 64 bytes of header, the final segment size is 576 bytes, which gives us a final file size of $576 \cdot 8,388,608 = 4,831,838,208$ bytes, or about 38,654.71 Megabits. Transferring this file over a 156Mb link would take $\frac{38,654.71\;Mb}{156\;Mb} \approx \boxed{247.79\;s}$

\hrulefill
\bigskip

\noindent{}2)
Let $SRTT = SampleRTT = \{106, 120, 140, 90, 155\}\\\\
ERTT_i = EstimatedRTT\\
DRTT_i = DevRTT\\
ToI_i = TimeoutInterval\\
$
(Where $i$ represents each iteration, IE. $i=0$ represents initial conditions.)\\

\noindent$ERTT_0 = 100\\
\alpha = 0.125\\
\beta = 0.25\\$

\noindent
To get the first estimated RTT:
\begin{align*}
ERTT_n &= \alpha \cdot SRTT_n + (1-\alpha)\cdot ERTT_{n-1}\\
ERTT_1 &= \alpha \cdot SRTT_1 + (1-\alpha)\cdot ERTT_0\\
&= 0.125 \cdot 106 + 0.875 \cdot 100\\
&= 13.25 + 87.5\\
ERTT_1 &= \boxed{100.75}
\end{align*}
\noindent
To get the first dev RTT:
\begin{align*}
DRTT_n &= \beta \cdot \lvert SRTT_n - ERTT_n \rvert + (1-\beta) \cdot DRTT_{n-1} \\
DRTT_1 &= \beta \cdot \lvert SRTT_1 - ERTT_1 \rvert + (1-\beta) \cdot DRTT_0 \\
&= 0.25 \cdot \lvert 106 - 100.75 \rvert + (1 - 0.25) \cdot 5 \\
&= 0.25 \cdot 5.25 + 0.75 \cdot 5 \\
&= 1.3125 + 3.75\\
DRTT_1 &= \boxed{5.0625\;ms} 
\end{align*}
\noindent
And to get the first Timeout Interval:
\begin{align*}
ToI_n &= ERTT_n + 4 \cdot DRTT_n\\
ToI_1 &= ERTT_1 + 4 \cdot DRTT_1\\
&= 100.75 + 4 \cdot 5.0625\\
ToI_1 &= \boxed{121\;ms}
\end{align*}

\noindent{}From here the calculation can be repeated for each of the other 5 values given for Sample RTT, using the values calculated from the previous iteration. The results are as follows: 

\begin{align*}
ERTT_2 &= 103.156\;ms \\
ERTT_3 &= 107.762\;ms \\
ERTT_4 &= 105.542\;ms \\
ERTT_5 &= 106.724\;ms \\\\
DRTT_2 &= 8\;ms \\
DRTT_3 &= 14.06\;ms \\
DRTT_4 &= 14.431\;ms \\
DRTT_5 &= 12.892\;ms \\\\
ToI_2 &= 135.156\;ms \\
ToI_3 &= 164\;ms \\
ToI_4 &= 163.266\;ms \\
ToI_5 &= 158.292\;ms 
\end{align*}

\pagebreak
\noindent
3) Subnet 1: 60 + 2 \textless\ 64\\
Subnet 2: 90 + 2 \textless\ 128\\
Subnet 3: 12 + 2 \textless\ 16\\
\noindent\\
Subnet 1 $\rightarrow$ 223.1.17.10?????? (6 bits for $\leq\;$63 addr) $\rightarrow$ \fbox{223.1.17.128/26}\\
Subnet 2 $\rightarrow$ 223.1.17.0??????? (7 bits for $\leq\;$127 addr)$\rightarrow$ \fbox{223.1.17.0/25}\\
Subnet 3 $\rightarrow$ 223.1.17.1100???? (4 bits for $\leq\;$15 addr)$\rightarrow$ \fbox{223.1.17.192/28}



\end{document}