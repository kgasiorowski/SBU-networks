\documentclass[12pt]{report}
\usepackage{amsmath}
\usepackage{ragged2e}
\setlength\parindent{0pt}

\begin{document}

\Large
\centering
CSE310 Written Homework \#2

\justify
\normalsize

Kuba Gasiorowski\\
ID: 109776237\\

\noindent{}1a) $2^{8\cdot4} = 2^{32} = 4,294,967,296$ possible sequence numbers. MSS is irrelevant since the sequence number is incremented by the number of bytes sent. So, $\boxed{L\approx4.294\;GB}$

\bigskip

\noindent{}1b)

\bigskip

\noindent{}2)
Let $SRTT = SampleRTT = \{106, 120, 140, 90, 155\}\\\\
ERTT_i = EstimatedRTT\\
DRTT_i = DevRTT\\
ToI_i = TimeoutInterval\\
$
(Where $i$ represents each iteration, IE. $i=0$ represents initial conditions.)\\

\noindent$ERTT_0 = 100\\
\alpha = 0.125\\
\beta = 0.25\\$

\noindent
To get the first estimated RTT:
\begin{align*}
ERTT_1 &= \alpha \cdot SRTT_1 + (1-\alpha)\cdot ERTT_0\\
&= 0.125 \cdot 106 + 0.875 \cdot 100\\
&= 13.25 + 87.5\\
ERTT_1 &= \boxed{100.75}
\end{align*}
\noindent
To get the first dev RTT:
\begin{align*}
DRTT_1 &= \beta \cdot \lvert SRTT_1 - ERTT_1 \rvert + (1-\beta) \cdot DRTT_0 \\
&= 0.25 \cdot \lvert 106 - 100.75 \rvert + (1 - 0.25) \cdot 5 \\
&= 0.25 \cdot 5.25 + 0.75 \cdot 5 \\
&= 1.3125 + 3.75\\
DRTT_1 &= \boxed{5.0625\;ms} 
\end{align*}
\noindent
And to get the first Timeout Interval:
\begin{align*}
ToI_1 &= ERTT_1 + 4 \cdot DRTT_1\\
&= 100.75 + 4 \cdot 5.0625\\
ToI_1 &= \boxed{121\;ms}
\end{align*}

\noindent{}From here the calculation can be repeated for each of the other 5 values given for Sample RTT. The results are as follows: \\\\
$
ERTT_2 = 103.156\;ms \\
ERTT_3 = 107.762\;ms \\
ERTT_4 = 105.542\;ms \\
ERTT_5 = 106.724\;ms \\\\
DRTT_2 = 8\;ms \\
DRTT_3 = 14.06\;ms \\
DRTT_4 = 14.431\;ms \\
DRTT_5 = 12.892\;ms \\\\
ToI_2 = 135.156\;ms \\
ToI_3 = 164\;ms \\
ToI_4 = 163.266\;ms \\
ToI_5 = 158.292\;ms \\\\
$
3)

\end{document}